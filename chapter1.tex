\chapter{Introduction}
\label{introchap}

\lewis{X}{1}{2}{3}{4}{5}{6}{7}{8}

Disaster events observed in recent years have been characterized by a strong element of digital, public participation.  While public involvement in disaster events is not a new phenomenon, information communication technology (ICT) \cite{palen:ict} makes this involvement more visible and enables new types of volunteer activities for interested members of the public.  The field of crisis informatics \cite{palen:vision} is the study of how people use ICT in crisis; it examines socio-technical issues in emergency response, and considers the actions of both official responders as well as members of the public.  The actions of members of the public during disaster, and including the actions of digital volunteers \cite{starbird:voluntweeters} in disaster have been the focus of recent crisis informatics research.

The emergence of interested citizens participating in a form of remote volunteerism during a disaster (also known as digital volunteerism) presents an interesting opportunity to develop software systems which support these volunteers' work and to explore issues at the intersection of human and machine computation.  The purpose of this work is to describe one such system whose purpose is to utilize digital volunteers to facilitate pet-to-family reunification in disaster.  This system is built to incorporate and synthesize human computation, crowd work, and machine learning; this thesis is an exploration of various design, implementation, and evaluation issues that arise in the course of creating such a system.  In addition to describing the design and architecture of this system, this work presents an experimental evaluation of the system's machine learning component, which is designed to support and assist the activities of digital volunteers.

The remainder of this work is divided into four chapters.  The second chapter discusses related literature and the problem of pet-to-family reunification in disaster.  The third chapter presents the interface and system design of \nplh, a software system which facilitates pet-to-family reunification.  The fourth chapter details the machine learning components of this system and presents an evaluation of this system in an experimental setting.  The fifth chapter concludes this work by reviewing key ideas and findings in each section, reflecting on the work as a whole, and presenting directions for future research.

