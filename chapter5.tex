\chapter{Conclusion}
\label{conclusion}

This chapter reviews the work presented in this thesis, summarizing the key points from each chapter, reflecting on the results of this work and its contributions, and discussing future work.

\section {Summary}

Chapter 2 explored the background behind this work, discussing the problem domain of pet-to-family reunification as well as relevant literature which informs the rest of the thesis.  Displaced pets are a large problem in disaster situations, giving rise to various public safety and health issues.  Furthermore, single individuals are ill-suited to the task of locating a specific pet for a variety of reasons, the most prominent being the potential size of the search space.  Utilizing digital volunteerism to facilitate crowd work to help address this problem mitigates this problem; creating a specialized system for digital volunteers to perform this work mitigates some of the challenges facing digital volunteers.  There is also space for machine learning systems to aid in the completion of this task.  Because the task is inherently difficult, the approach taken in this work integrates human computation with machine computation, with the latter assuming the paradigm of an agent collaborating with users.

Chapter 3 presents the system design and architecture of \nplh{}, a software platform for facilitating pet-to-family reunification.  The system's interface and architectural design are influenced primarily by the envisioned tasks of volunteers: reporting pets, matching lost pet reports with found pet reports, editing pet reports, and verifying proposed matches.  Motivated by a diverse user demographic and the need to provide a simple yet powerful interface for volunteers to complete these tasks, a key feature of this system is an accessible user interface, which is an area of ongoing collaborative work \cite{sdc}.  Collaboration mechanisms and social capital constitute the other key design features of \nplh.  The software architecture of the system is a standard client-server model; data is stored in a document-oriented store, and the software features of the application are implemented as Model-View-Controller Django applications.

Chapter 4 discusses the design and evaluation of potential implementations of the machine learning components of \nplh.  An information retrieval system, implemented in Lucene, is used as a backbone for performing ranked retrieval of pet reports corresponding to a given pet, which is the essential functionality used by the {\tt matching} application.  Two classifiers are designed and evaluated; the first is the {\tt Pet} classifier, which attempts to model the quality of a pet report, and the second is the {\tt Match} classifier, which attempts to model the correlations between successful pet matches.  Various implementations of these classifiers (Naive Bayes, Nearest Neighbors, and Decision Tree algorithms) are individually evaluated.  The labeled data used for each evaluation was generated by annotators working off of a raw corpus (obtained from a public database of lost and found pet reports), performing annotation tasks that closely mirror predicted digital volunteer activities.  The evaluation of each classifier revealed that Naive Bayes implementations performed the best.  In an end-to-end evaluation of the system, various scoring functions (which incorporated the two classifiers to different extents) were used to re-rank baseline Lucene rankings generated for 101 queries that included a true positive (i.e., the query was a pet that had a matching report generated by an annotator).  This evaluation showed that the {\tt Pet} classifier was generally ineffective at providing information useful to re-ranking results and identifying the true positive match, whereas the {\tt Match} classifier was effective at improving the rank of the true positive over the result set.

\section {Reflection and Contributions}

One of the primary contributions of this work is the design and architecture of \nplh; this effort is informing the ongoing implementation of the system in preparation for a public deployment in a real disaster setting.  Furthermore, the evaluations performed in Chapter 4 are valuable in assessing the worth of the approach taken in this work, which is to combine human computation and machine computation in a symbiotic manner to better approach the problem of pet-to-family reunification.  Although the original scoring system, which incorporates machine learning measures of pet report quality and match likelihood (the {\tt Pet} and {\tt Match} classifiers, respectively), was relatively unsuccessful, the other explorations performed in this work reveal that even a simple Lucene implementation combined with the {\tt Match} classifier hold great promise in fulfilling the role of a machine learning collaborator which observes and supports the human computation being done by digital volunteers as a crowd.

\section {Future Work}

The most apparent need for future work on this system is a public deployment of the system in an actual disaster scenario.  This would permit the study of the various social mechanisms proposed as system features in addition to providing an opportunity to study the machine learning components of the system in a more realistic setting.  Furthermore, a system deployment would permit the design and evaluation of potentially interesting features that could be included in the machine learning components of the system but were infeasible to simulate and evaluate in the experimental setting presented in Chapter 4.  For example, it would be interesting to consider location correlations between pet reports (I envision this being done by using some kind of stratification via clustering), or correlation between the sets of users who have submitted or worked on pet reports as valuable features for the {\tt Match} classifier.  

Furthermore, more work is needed to continue to develop the machine learning components of the system in general.  The exploratory implementations and evaluations of the classification algorithms presented in Chapter 4 are simply platforms for future iteration and refinement.  In addition, the core information retrieval system used in Chapter 4 was deliberately simple in its implementation (because the focus of the evaluation was on the classification components); more work in tuning this component would undoubtedly result in higher performance.  Even radically different components could be considered for future development of the machine learning system; the primary purpose of this work's implementations and evaluations was to assess the potential improvements that a machine learning system can offer to the problem of pet-to-family reunification.
